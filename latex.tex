\documentclass[11pt, oneside]{article}   	% use "amsart" instead of "article" for AMSLaTeX format
\usepackage{geometry}                		% See geometry.pdf to learn the layout options. There are lots.
\usepackage{listings, jlisting}
\usepackage[version=3]{mhchem}
\renewcommand{\lstlistingname}{リスト}
\lstset{
%language = R,
%language = C++,   
breaklines = true,
numbers = left,
frame = tbrl,
tabsize = 4,
basicstyle =\ttfamily,
captionpos = t
}
\geometry{letterpaper}                   		% ... or a4paper or a5paper or ... 
%\geometry{landscape}                		% Activate for rotated page geometry
%\usepackage[parfill]{parskip}    		% Activate to begin paragraphs with an empty line rather than an indent
\usepackage{graphicx}				% Use pdf, png, jpg, or eps§ with pdflatex; use eps in DVI mode
								% TeX will automatically convert eps --> pdf in pdflatex		
\usepackage{amssymb}

%SetFonts

%SetFonts


\title{\LaTeX}
\author{Akihiro Minamino}
%\date{}							% Activate to display a given date or no date

\begin{document}
\maketitle

\section{スタイルファイル}
\subsection{listings.styとjlisting.sty}
詳しくは、
\verb| http://xyoshiki.web.fc2.com/tex/listings.html |を参照。\\
 \\
使用例
\begin{lstlisting}
\usepackage{listings, jlisting}
\renewcommand{\lstlistingname}{リスト}
\lstset{
%language = R,
%language = C++,   
breaklines = true,
numbers = left,
frame = tbrl,
tabsize = 4,
basicstyle =\ttfamily,
captionpos = t
}

\begin{lstlisting}
ソースコード
バックスラッシュ end{lstlisting}
\end{lstlisting} 
実行例
\begin{lstlisting}
ソースコード
\end{lstlisting}
 \\
オプション(全体)の説明
\begin{lstlisting}
language= lstlisting環境内の言語の指定。参照
numbers=	 行番号表示
  デフォルト:none
  他のオプション:left、right
stepnumber= 行番号増分
numberstyle= 行番号の書体指定
numbersep= 行番号と本文の間隔
  デフォルト:10pt。
breaklines= 行が長くなってしまった場合の改行
  デフォルト:false
  他のオプション:true
breakindent= 改行時インデント量
  デフォルト:20pt。
frame= frameの指定
 デフォルト:none
 他のオプション:leftline、topline、bottomline、lines、single、shadowbox
framesep= frameまでの間隔
basicstyle= 書体の指定
  おすすめの書体: \ttfamily と \small\ttfamily
commentstyle=	 注釈の書体
keywordstyle= キーワードの書体指定
caption= キャプションの指定
label= ラベルの指定
\end{lstlisting}


\subsection{mhchem.sty}
化学式の表現に利用。\\
詳しくは、
\verb| https://doratex.hatenablog.jp/entry/20131203/1386068127 |を参照。\\
基本となるのが\verb|\ce|という命令である。
 \\
使用例
\begin{lstlisting}
\usepackage[version=3]{mhchem}

\ce{_{17}^{35}Cl}
\end{lstlisting} 
実行例\\
\ce{_{17}^{35}Cl}



\end{document}  